\documentclass{article}
\usepackage{graphicx} % Required for inserting images
\usepackage{amsmath}
\usepackage{hyperref}
\title{Optimization of Magnetic Reference Layer Design in Polarized Neutron Reflectometry}
\author{Michael Birtman}
\date{October 2025}
 
\begin{document}

\maketitle


\section*{Background}
\paragraph{ALL INFO CURRENTLLY FROM COMPANY PAPER: }
\noindent
\newline
Polarized neutron reflectometry measures how neutrons reflect from a layered sample and provides two reflectivity curves (spin up and spin down). So each neutron spin state has a different SLD, the difference means that by introducing an MRL we get a difference in the reflectivity between different neutron spin channels. The reflectivity is governed by SLD variations at interfaces (WHY?). 

\section*{Baseline Solution}
Since I will experiment with solutions, I think the first solution to gain insight on the problem is by implementing a simple "baseline" solution. This will be done by simply finding the optimum for the strucuture given at the paper [bla bla bla]. This would be so that we can see if the further experimentation on the structural approach is known to better or worse than the baseline solution that is presented in the paper. In part we'll simply solve for the optimal without looking at different materials. 
\newline
\paragraph{Basic}
Let $x = (x_{CoTi}, d_{MRL}, c_{material})$, where 
\begin{itemize}
    \item $x_{CoTi}$ is the alloy composition 
    \item $d_{MRL}$ is the MRL layer thickness 
    \item $c_{mateiral}$ is the capping layer material
\end{itemize}
We want to solve with respect to some metric, using the propsed metric in the paper we get: 

${\rm TSF}(x_{\text{CoTi}},\; d_{\text{MRL}},\; c)$
$x_{\text{CoTi}}\in [0,1]$, $d_{\text{MRL}} > 0$, $c\in{\text{Al}_2\text{O}_3,\;
\text{SiO}_2,\text{Au}}$. 

\paragraph{Basic 2}
We can further introduce more parameters and optimize them. Some of these were static in the paper. So we can do better. Let's define an extended version of $x = (x_{Co}, d_{MRL}, \sigma_{MRL/sub}, \sigma_{MRL/cap}, d_{cap}, \sigma_{cap}, B, \lambda, b)$. 

So we can define a an objective function as: given some SOI $t \in T$. Let's first define the functions that build the sensitivty function. 
\begin{itemize}
    \item $R_s^{(t)}(Q;x)$ simulated polarized refdlectivity of the full stack 
    \item $R_{s, sub}(Q;x)$ same stack without the SOI s
\end{itemize}
We then define the function the senstivty function $S_s^{(t)} = \frac{R_{s;sub}^{(t)}(Q;x) - R_s^{(t)}(Q;x)}{R_{s,sub}^{(t)}(Q;x)-R_s^{(t)}(Q;x)}$. Where $Q\in [Q_{min}, Q_{max}]$ \cite{main}. So now we can define the 
$$
\mathrm{TSF}(x)
= 
\sum_{t \in \mathcal{T}}
\int_{Q_{\min}}^{Q_{\max}}
w(Q)\,
\Big(
\big| S^{(t)}_{\uparrow}(Q; x) \big|
+
\big| S^{(t)}_{\downarrow}(Q; x) \big|
+
\big| S^{(t)}_{\uparrow}(Q; x) - S^{(t)}_{\downarrow}(Q; x) \big|
\Big)
\, dQ.$$ I have added an optinal weight function $w(Q) \equiv1$, but we can define a discrete function or something else so we can prioritize some parts of Q over other parts. We can also break the function down into it's parts getting two functions the ones that together build $\rm TSF$ the point would be to have control over the prioriy of the optimization and build a Pareto optimization and then analyze the priority of the problem. We would do that by simply looking at this. Then we could weight the differnt function such that the solution prioritizes functions differently. From there we could perhaps gain insight to the optimzation problem. 
$$
\max_{x} \big( \mathrm{SFM}_{\mathrm{tot}}(x),\; \mathrm{MCF}(x) \big)
$$





\subsection*{Optimzation method}
To correctly choose the optimzation method we're going to do, we need to understand the landscape of the different functions as well as the computational load each computation of TSF is. This will give insight in what method of optimzation we ought to choose. 

\section*{Defining the Optimization Problem} 
\paragraph{SOURCES: MAINLY MATH BOOKS ON VAR CALC AND OWN REASONIG: }

\noindent
\newline
There are multiple levels of optimization we can apply to this problem. For example, simply constructing an objective function given by \href{} and some restraints. However, we can do better. Firstly there are multiple \emph{objective functions} to look at, some of which are going in different directions. We can also mix different materials, first part will be going through the theory of variational calculus, where we essentially optimizing a function that represents the structure of the thin film. We can do this by applying calculus of variation theory. Here we'd define a functional optimization problem where we can optimize the structure of the thin film. 

We will also be optimizing for the parameters of the structure, let's say the width, the material, the mixture of the material, the cost etc. We will also provide some Pareto optimization techniques such that trade offs of different solutions can be visualized by the experts. Different robustness techniques will be used to test the different peaks and see their resilience to errors in construction. 

In this section we will look at the problem in detail, and derive and create a optimization problem given the problem structure. As the problem heavily depends on the physics of the problem we might be able to look into different fundamental structures of the thin film to improve the results. In this case we need to build a working optimization function that will be able to find stable structure that produce good results given the problem. 


\paragraph{Performance}
We have to define some metric function and we can do that by defining a spin contrast $C(Q; x(z))$, simply a function that measure spin reflectivity moment transfer. We can be creative when defining this function by looking at the difference, log difference, normalize it etc. Currently I'm not at all sure about how we ought to define this function. 

A basic definition could be: 
\[
J_{performance}[x(z)] = \int_{Q_{min}}^{Q_{max}} C(Q;x(z))w(Q)dQ
\]
where $w(Q)$ is simply a weight function to prioritize some Q values, the relevance of this depends on the physics and perhaps this SOI. Currently I'm not sure about how useful this part is. 

We can also define the \emph{simplest choice} where we just have $C(Q; x(z)) = |R_{\uparrow}(Q) - R_{\downarrow}(Q)|$. This is also of course a non linear optimization problem and defining a Lagrangian to optimize instead may be useful given the structure. Observe the current structure: 

\[
\mathcal{F}[x(z)] = J_{performance} - \lambda_s \int_0 ^L x'(z)^2dz - \lambda_c \int_0 ^LV(x(z))dz - \lambda_r \int_{Q_{min}}^{Q_{max}}D(R(Q;x(z), Q)dQ, \ \lambda_i > 0 
\]
Where, we can define a reflectivity distortion function. There are many ways to define this functional, and depending on what exactly is relevant we may want to change it. 


\section*{}



\end{document}